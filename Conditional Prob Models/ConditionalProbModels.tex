%% LyX 2.2.3 created this file.  For more info, see http://www.lyx.org/.
%% Do not edit unless you really know what you are doing.
\documentclass[ruled]{article}
\usepackage{courier}
\usepackage[T1]{fontenc}
\usepackage[latin9]{inputenc}
\usepackage[letterpaper]{geometry}
\geometry{verbose}
\usepackage{color}
\usepackage{url}
\usepackage{algorithm2e}
\usepackage{amsmath}
\usepackage{amssymb}
\usepackage{titlesec}
\newcommand{\sectionbreak}{\clearpage}
\usepackage{graphicx}
\usepackage[export]{adjustbox}
\graphicspath{ {./images/} }
\usepackage{listings}
\usepackage{pdfpages}
\usepackage[unicode=true,
 bookmarks=false,
 breaklinks=false,pdfborder={0 0 1},backref=section,colorlinks=true]
 {hyperref}

\makeatletter

%%%%%%%%%%%%%%%%%%%%%%%%%%%%%% LyX specific LaTeX commands.
\providecommand{\LyX}{\texorpdfstring%
  {L\kern-.1667em\lower.25em\hbox{Y}\kern-.125emX\@}
  {LyX}}
%% Special footnote code from the package 'stblftnt.sty'
%% Author: Robin Fairbairns -- Last revised Dec 13 1996
\let\SF@@footnote\footnote
\def\footnote{\ifx\protect\@typeset@protect
    \expandafter\SF@@footnote
  \else
    \expandafter\SF@gobble@opt
  \fi
}
\expandafter\def\csname SF@gobble@opt \endcsname{\@ifnextchar[%]
  \SF@gobble@twobracket
  \@gobble
}
\edef\SF@gobble@opt{\noexpand\protect
  \expandafter\noexpand\csname SF@gobble@opt \endcsname}
\def\SF@gobble@twobracket[#1]#2{}

\@ifundefined{date}{}{\date{}}
%%%%%%%%%%%%%%%%%%%%%%%%%%%%%% User specified LaTeX commands.
\definecolor{mygreen}{rgb}{0,0.6,0}
\definecolor{mygray}{rgb}{0.5,0.5,0.5}
\definecolor{mymauve}{rgb}{0.58,0,0.82}

\makeatother

\usepackage{listings}
\lstset{backgroundcolor={\color{white}},
basicstyle={\footnotesize\ttfamily},
breakatwhitespace=false,
breaklines=true,
captionpos=b,
commentstyle={\color{mygreen}},
deletekeywords={...},
escapeinside={\%*}{*)},
extendedchars=true,
frame=shadowbox,
keepspaces=true,
keywordstyle={\color{blue}},
language=Python,
morekeywords={*,...},
numbers=none,
numbersep=5pt,
numberstyle={\tiny\color{mygray}},
rulecolor={\color{black}},
showspaces=false,
showstringspaces=false,
showtabs=false,
stepnumber=1,
stringstyle={\color{mymauve}},
tabsize=2}

 
\begin{document}

\global\long\def\reals{\mathbf{R}}
 \global\long\def\integers{\mathbf{Z}}
\global\long\def\naturals{\mathbf{N}}
 \global\long\def\rationals{\mathbf{Q}}
\global\long\def\ca{\mathcal{A}}
\global\long\def\cb{\mathcal{B}}
 \global\long\def\cc{\mathcal{C}}
 \global\long\def\cd{\mathcal{D}}
\global\long\def\ce{\mathcal{E}}
\global\long\def\cf{\mathcal{F}}
\global\long\def\cg{\mathcal{G}}
\global\long\def\ch{\mathcal{H}}
\global\long\def\ci{\mathcal{I}}
\global\long\def\cj{\mathcal{J}}
\global\long\def\ck{\mathcal{K}}
\global\long\def\cl{\mathcal{L}}
\global\long\def\cm{\mathcal{M}}
\global\long\def\cn{\mathcal{N}}
\global\long\def\co{\mathcal{O}}
\global\long\def\cp{\mathcal{P}}
\global\long\def\cq{\mathcal{Q}}
\global\long\def\calr{\mathcal{R}}
\global\long\def\cs{\mathcal{S}}
\global\long\def\ct{\mathcal{T}}
\global\long\def\cu{\mathcal{U}}
\global\long\def\cv{\mathcal{V}}
\global\long\def\cw{\mathcal{W}}
\global\long\def\cx{\mathcal{X}}
\global\long\def\cy{\mathcal{Y}}
\global\long\def\cz{\mathcal{Z}}
\global\long\def\ind#1{1(#1)}
\global\long\def\pr{\mathbb{P}}

\global\long\def\ex{\mathbb{E}}
\global\long\def\var{\textrm{Var}}
\global\long\def\cov{\textrm{Cov}}
\global\long\def\sgn{\textrm{sgn}}
\global\long\def\sign{\textrm{sign}}
\global\long\def\kl{\textrm{KL}}
\global\long\def\law{\mathcal{L}}
\global\long\def\eps{\varepsilon}
\global\long\def\convd{\stackrel{d}{\to}}
\global\long\def\eqd{\stackrel{d}{=}}
\global\long\def\del{\nabla}
\global\long\def\loss{\ell}
\global\long\def\tr{\operatorname{tr}}
\global\long\def\trace{\operatorname{trace}}
\global\long\def\diag{\text{diag}}
\global\long\def\rank{\text{rank}}
\global\long\def\linspan{\text{span}}
\global\long\def\proj{\text{Proj}}
\global\long\def\argmax{\operatornamewithlimits{arg\, max}}
\global\long\def\argmin{\operatornamewithlimits{arg\, min}}
\global\long\def\bfx{\mathbf{x}}
\global\long\def\bfy{\mathbf{y}}
\global\long\def\bfl{\mathbf{\lambda}}
\global\long\def\bfm{\mathbf{\mu}}
\global\long\def\calL{\mathcal{L}}
\global\long\def\vw{\boldsymbol{w}}
\global\long\def\vx{\boldsymbol{x}}
\global\long\def\vxi{\boldsymbol{\xi}}
\global\long\def\valpha{\boldsymbol{\alpha}}
\global\long\def\vbeta{\boldsymbol{\beta}}
\global\long\def\vsigma{\boldsymbol{\sigma}}
\global\long\def\vmu{\boldsymbol{\mu}}
\global\long\def\vtheta{\boldsymbol{\theta}}
\global\long\def\vd{\boldsymbol{d}}
\global\long\def\vs{\boldsymbol{s}}
\global\long\def\vt{\boldsymbol{t}}
\global\long\def\vh{\boldsymbol{h}}
\global\long\def\ve{\boldsymbol{e}}
\global\long\def\vf{\boldsymbol{f}}
\global\long\def\vg{\boldsymbol{g}}
\global\long\def\vz{\boldsymbol{z}}
\global\long\def\vk{\boldsymbol{k}}
\global\long\def\va{\boldsymbol{a}}
\global\long\def\vb{\boldsymbol{b}}
\global\long\def\vv{\boldsymbol{v}}
\global\long\def\vy{\boldsymbol{y}}


\title{Machine Learning and Computational Statistics\\
Conditional Probability Models}

\maketitle
\textbf{Intro}: This document consists of concepts and exercises related to Conditional Probability Model. The core idea lies in the frequentist's methodology in which parameters (e.g., $\theta$) is not a random variable. \\ 

Learning objectives of  Conditional Probability Models include: \\

\begin{enumerate}
\item Point estimation 
	\begin{enumerate}
	\item One point of statistical problem is point estimation
	\item A statistic s = s($D$) is any function of the data
	\item A statistic $\hat{\theta}$ ($D$) taking values in $\Theta$ is a point estimator of $\theta$
	\item A good point estimator will have $\hat{\theta} \approx \theta$
	\item Desirable statistical properties of point estimators are: 
		\begin{enumerate}
			\item Consistency: As data size $n \rightarrow \inf$, we get $\hat{\theta}_n \rightarrow \theta$.
			\item Efficiency: $\hat{\theta}_n$ is as accurate as we can get from a sample of size n.
		\end{enumerate}
	\item Maximum likelihood estimators (MLE) are consistent and efficient under reasonable conditions.
	\end{enumerate}
	
\item Common approaches to getting a point estimator for $\theta$
	\begin{enumerate}
		\item Maximum Likelihood (MLE)
				\begin{enumerate}
				\item The likelihood of an estimate of a probability distribution for some data $D$. 
\[
p(D | \theta) = \prod_{i=1}^{n} p(y_i | \theta)
\]
\[L_D(\theta) = log p(D | \theta)\]

				\item The MLE for some parameter $\theta$ of a probability model. 
\[
\hat{\theta}_{MLE} = \underset{\theta \in \Theta}{\mathrm{argmax}} L_D (\theta) 
\]
				\item Describe the basic structure of a linear probabilistic model, in terms of 1. a parameter $\theta$ of the probablistic model, 2. a linear score function, 3. a transfer function (kin to a "response function" or "inverse link" function, though we have relaxed requirements on the parameter theta).
				\item We can use MLE to find $w$ by represent the parameter (e.g., $\theta$) as a function of $x, w$ and differentiate log-likelihood of the set-up probabilistic model to find $w$.
				\item Common transfer functions for some distributions and why: 
					   \begin{enumerate}
					   \item bernoulli
					   \item poisson: $\lambda = \psi(w^Tx) = \exp^{w^Tx}$
					   \item gaussian: $y | x \sim \mathcal{N}(w^Tx, \sigma^2)$
					   \item categorical distributions: $\psi(s) = \psi(s_0, s_1, ... s_n) =( \frac{\exp^s_0}{\sum_{i=1}^{n} \exp^{s_i}}, \frac{\exp^s_1}{\sum_{i=1}^{n} \exp^{s_i}}, ..., \frac{\exp^s_n}{\sum_{i=1}^{n} \exp^{s_i}}   )$
					   \end{enumerate}  
				\end{enumerate}
	 \item Method of Moment
	 \item Maximum A Posterior (MAP) (Bayesian method)
	 \[
		P(\theta | D) = \frac{P(D|\theta) * P(\theta)}{P(D)} \propto P(D|\theta) * P(\theta)
		\]

      \[
      \begin{split}
      \hat{\theta}_{MAP} &= \underset{\theta}{\mathrm{argmax}}P(D|\theta) * P(\theta) \\
      &= \underset{\theta}{\mathrm{argmax}} log P(D|\theta) * P(\theta) \\
      &= \underset{\theta}{\mathrm{argmax}} log \prod_{i=1}^{n} P(y_i | x_i, w) * P(\theta) \\
      &= \underset{\theta}{\mathrm{argmax}}  \sum_{i=1}^{n} log P(y_i | x_i, w) * P(\theta) \\
      \end{split}
      \]     

	\end{enumerate}

\item \href{https://wiseodd.github.io/techblog/2017/01/01/mle-vs-map/}{MLE vs. MAP}
\item Concept check questions \href{https://davidrosenberg.github.io/mlcourse/ConceptChecks/10-Lab-Check_sol.pdf}{Conditional Probability Models}   

%\includepdf[pages=28-29,pagecommand={},width=\textwidth]{08a-bayesian-regression}

\end{enumerate}



\end{document}
